\documentclass[12pt]{article}
\usepackage[utf8]{inputenc}
\usepackage{amsmath,amsthm,amsfonts,amssymb}
\usepackage{tikz}
%\usepackage{subfig}
\usepackage[english]{babel}
%\usepackage{capt-of}
\newtheorem{theorem}{Theorem}
\usetikzlibrary{calc}
\usetikzlibrary{shapes}
\usepackage{hyperref}
%might be unnecessary
\usepackage{doi}
%bibliography CMDS


\usepackage{pdflscape}

\usepackage{filecontents}
\begin{filecontents*}{ohcrefs.bib}

@article{klarner_1967, 
title={Cell Growth Problems}, 
volume={19}, DOI={10.4153/CJM-1967-080-4}, 
journal={Canadian Journal of Mathematics}, 
publisher={Cambridge University Press}, 
author={Klarner, David A.}, 
year={1967}, 
pages={851–863}}

\end{filecontents*}

\usepackage[style=alphabetic]{biblatex}
\addbibresource{ohcrefs.bib}

%%% With amsthm package, creates environments for nicely formatted,
%%% labeled, and numbered propositions, etc.
\theoremstyle{plain}
\newtheorem{thm}{Theorem}
\newtheorem{lemma}[thm]{Lemma}
\newtheorem{prop}[thm]{Proposition}
\newtheorem{conj}[thm]{Conjecture}
\newtheorem{cor}[thm]{Corollary}
\newtheorem{claim}[thm]{Claim}
\newtheorem{fact}[thm]{Fact}

\theoremstyle{definition}
\newtheorem{eg}[thm]{Example}
\newtheorem{defn}[thm]{Definition}
\newtheorem{rem}[thm]{Remark}
\newtheorem{observ}[thm]{Observation}
\newtheorem{open}[thm]{Open Problem}
\newtheorem{prob.}[thm]{Problem}
\newtheorem{quest}[thm]{Question}

% I used these for making definitions and theorems, not what is above
\theoremstyle{remark}
\newtheorem{remark}[thm]{Remark}
\newtheorem{note}[thm]{Note}
\theoremstyle{definition}
\newtheorem{definition}{Definition}[section]
\newtheorem{exmp}{Example}[section]

%nice quick solution
\usepackage[margin=1in]{geometry}

%doc info
\title{Symmetric Subsequences from the Left}
\author{Jack Hanke}
\date{\today}

\begin{document}

\maketitle

Consider a sequence $s = (s_1, \dots, s_n)$ where $s_i \in \mathbb{Z}_b$. We are interested in the function $f: s \to \mathbb{Z}^+$ defined as follows. Find the largest $i$ so that $(s_1, s_2, \dots, s_i)$ is symmetric. Continue by finding the largest $j$ so that $(s_{i+1}, s_{i+2}, \dots, s_{j})$ is symmetric. Repeat until $s_n$ is part of a symmetric subsequence. Also let $g$ return the sequence of lengths of symmetric subsequences in $s$. For example, $f((1,1,0,1,0,0)) = 3$ and $g((1,1,0,1,0,0)) = (2,3,1)$, as the process yields $(1,1)$, $(0,1,0)$, and $(0)$.

Next, let the addition of sequences $s = (s_1, \dots, s_n)$ and $t = (t_1, \dots, t_m)$ to be $s+t = (s_1,\dots,s_n,t_1,\dots,t_m)$. Note in general that $f(s+t) \neq f(t+s)$. Finally, let 

$$m_{b,n} = \max_{s,b} f(s).$$

\begin{theorem}
    $m_{b,n}=n$ for $b>2$
\end{theorem}

\begin{proof}
    Construct the sequence $s = (s_1, \dots, s_n)$ with $s_i = i \mod b$. This sequence has the maximum number of symmetric subsequences $n$. First, notice the values equivalent to $s_1$ are $s_{bk+1}$ for integer $k$ so that $bk+1 \leq n$, and these values are the only candidates to check for possible symmetric subsequences for $s_1$. However, the subsequence between $s_{bk+1}$ and $s_{b(k+1)+1}$ cannot be symmetric for $b>2$, as $(i+1) \mod b \neq (i-1) \mod b$ unless $b=2$. As the $(i+1) \mod b \neq (i-1) \mod b$ holds for all $i$, every $s_i$ must be a member of a length $1$ symmetric subsequence.
\end{proof}

Now consider the $b=2$ case.

\begin{lemma}\label{ones at end}
    A length $1$ symmetric subsequence in position $i$ implies $s_{i+1}=s_{i+2}=\dots=s_n$.
\end{lemma}

\begin{proof}
    Suppose there exists a length $1$ symmetric subsequence in position $i$. Then suppose there exists a value $j$ greater than $i$ so that $s_{j}=s_i$. Therefore $(s_{i+1},\dots,s_{j-1})$ must all be the opposite value to $s_i$. However, $(s_{i},\dots,s_{j})$ form a symmetric subsequence. Therefore there can exist no such $j$. 
\end{proof}

\begin{lemma}\label{no three pairs}
    There cannot exist three consecutive pairs of length $2$ symmetric sequences.
\end{lemma}

\begin{proof}
    Suppose there exists a subsequence $s_{i}, s_{i+1}, s_{i+2}, s_{i+3}, s_{i+4}, s_{i+5}$ where the three consecutive pairs form symmetric subsequences.  This implies that $s_{i}= s_{i+1}$, $s_{i+2}= s_{i+3}$, and $s_{i+4}= s_{i+5}$. As $(s_{i}, s_{i+1})$ form a length $2$ symmetric subsequence, $s_{i}\neq s_{i+2}$, as otherwise $(s_{i}, s_{i+1} s_{i+2}, s_{i+3})$ would form a larger symmetric subsequence. Similarly, $s_{i+2}\neq s_{i+4}$. However, this gives $s_{i}=s_{i+4}$, which implies $(s_{i}, s_{i+1} s_{i+2}, s_{i+3}, s_{i+4}, s_{i+5})$ forms a symmetric sequence.
\end{proof}

\begin{theorem}
    $m_{2,n} \leq \lfloor\frac{3n+10}{7}\rfloor$
\end{theorem}

\begin{proof}
    The first seven values for $m_{2,n}$ are $1,2,2,3,3,4,4$. We will build maximal sequences using these base sequences to form the upper bound. 
    
    From Lemma \ref{ones at end} we know that the first symmetric subsequences of length $1$ always proceeds the final symmetric subsequence. Therefore, the placement of the first length $1$ symmetric subsequence to maximize the density of subsequences is the $n-1$ position, which leaves the final sequence with $2$ length $1$ symmetric subsequences. Therefore a sequence $s'$ that maximizes $f$ could have

    $$g(s') = (\dots,1,1).$$

    From Lemma \ref{no three pairs} we know that there cannot exist $3$ consecutive length $2$ symmetric subsequences. Therefore $s'$ must also have

    $$g(s') = (\dots,3,2,2,1,1).$$

    We can continue adding a subsequence $t$ that has $g(t)=(3,2,2)$ to the beginning of $s'$ for as long as we'd like. Truncating the start of $s'$ at the beginnings of each symmetric subsequence gives an upper bound for $m_{2,n}$ for $n=2,4,6 \mod 7$. Specifically, for $n=2 \mod 7$ we have $m_{2,n} \leq m_{2,2} + \frac{3n}{7}$, $n=4 \mod 7$ we have $m_{2,n} \leq m_{2,4} + \frac{3n}{7}$, and $n=6 \mod 7$ we have $m_{2,n} \leq m_{2,6} + \frac{3n}{7}$.

    To achieve an upperbound for $n=1,3,5 \mod 7$, consider $s''$, which has

    $$g(s'') = (\dots, 3,2,2,1).$$

    Similarly we can continue adding a subsequence $t$ that has $g(t)=(3,2,2)$ to the beginning of $s''$, and truncating the start of $s''$ at the beginnings of each symmetric subsequence gives the same form of an upperbound for $n=1,3,5 \mod 7$. Finally, for $n=0 \mod 7$, consider $s'''$, which has

    $$g(s''') = (\dots,2,3,2,1,1)$$

    For $s'''$ we can continue adding a subsequence $t$ that has $g(t)=(2,3,2)$ to the beginning of $s'''$, and truncating the start of $s'''$ at the beginnings of each symmetric subsequence gives the same form of an upperbound for $n=0,2,4 \mod 7$.

    This gives us the upper bound $\forall n$.
    
\end{proof}

\begin{lemma}\label{lemma: special sequence}
    The sequence $s^*=(0,0,1,1,0,1,0)$ has $g(s^*)=(2,2,3)$ and  
    $$f\left(\sum_{i=1}^k s^*\right) = \sum_{i=1}^k f(s^*) = 3k$$ 
    $\forall k \geq 1$.
\end{lemma}

\begin{proof}
    We must first verify that the largest symmetric subsequence containing $s_1$ in $S=\sum_{i=1}^k s^*$ is $(s_1,s_2)$. Notice that $s_i=0$ for $i=4,6 \mod 7$ cannot be the end of a symmetric subsequence, as $s_i=1$ for $i=3,5 \mod 7$. Similarly $s_i=2$ for $i=2 \mod 7$ has $(s_1,s_2,s_3)=(0,0,1)$ and $(s_{i-2},s_{i-1},s_{i}) = (0,0,0)$ for $i>2$. Finally, $s_i=0$ for $i=1 \mod 7$ has $(s_1,s_2,s_3,s_3)=(0,0,1,1)$ and $(s_{i-3},s_{i-2},s_{i-1},s_{i})=(0,1,0,0)$ for $i>1$. So $(s_1,s_2)$ is the largest symmetric subsequence containing $s_1$

    As $s_2$ is part of the first subsequence, we next prove the largest symmetric subsequence containing $s_3$ is $(s_3,s_4)$. For $i=3 \mod 7$, we have $(s_3,s_4,s_5,s_6) = (1,1,0,1)$ and $(s_{i-3},s_{i-2},s_{i-1},s_{i}) = (0,0,1,1)$. For $i=4,6 \mod 7$, we have $(s_3,s_4) = (1,1)$ and $(s_{i-1},s_i) = (0,1)$.

    As $s_4$ is part of the first subsequence, we next prove the largest symmetric subsequence containing $s_5$ is $(s_5,s_6,s_7)$. For $i=1,2 \mod 7$, we have $(s_5,s_6)=(0,1)$ and $(s_{i-1},s_i)=(0,0)$. For $i=5 \mod 7$, we have $(s_5,s_6,s_7)=(0,1,0)$ and $(s_{i-2},s_{i-1},s_{i}) = (1,1,0)$. For $i=0 \mod 7$, we have $(s_5,s_6,s_7,s_8)=(0,1,0,0)$ and $(s_{i-3},s_{i-2},s_{i-1},s_{i})=(1,0,1,0)$.

\end{proof}

\begin{lemma}\label{lemma: other special sequence}
    The sequence $s^{**}=(0,0,1,0,1,1,1)$ has $g(s^**) = (2,3,2)$ and 
    $$f\left(\sum_{i=1}^k s^{**}\right) = \sum_{i=1}^k f(s^{**}) = 3k$$ 
    $\forall k \geq 1$.
\end{lemma}

\begin{proof}
    Proof mirrors the structure of Lemma \ref{lemma: special sequence}.
\end{proof}

\begin{theorem}
    $m_{2,n} \geq \lfloor\frac{3n+10}{7}\rfloor$
\end{theorem}

\begin{proof}
    To prove the lower bound we simply construct sequences that satisfy the bound for certain residues mod $7$.

    First, let $s^*=(0,0,1,1,0,1,0)$. Then let $k$ be the smallest multiple of $7$ greater than or equal to $n$, and construct the sequence 

        $$S = \sum_{i=1}^k s^*.$$

    Truncating $S$ after $S_{n-6}, S_{n-4}, S_{n-2}$, and $S_{n-1}$ gives a sequence of length congruent to $i=1,3,5,6 \mod 7$ and contains $\frac{3}{7}(k-7) + m_{2,i}$ symmetric subsequences. 

    For $n=0 \mod 7$ construct 
        $$T = (0) + \sum_{i=1}^k s^*.$$

    We must verify that the largest symmetric subsequence containing $t_1$ is $(t_1, t_2, t_3)$. For $i = 1,6 \mod 7$ $(t_1,t_2)=(0,0)$ and $(t_i,t_{i+1} = (1,0))$. For $i = 2 \mod 7$ $(t_1,t_2,t_3)=(0,0,0)$ and $(t_i,t_{i+1},t_{i+2} = (1,0,0))$. For $i = 3 \mod 7$ $(t_1,t_2,t_3,t_4,t_5)=(0,0,0,1,1)$ and $(t_{i}, t_{i+1}, t_{i+2}, t_{i+3}, t_{i+4})=(0,1,0,0,0)$. Therefore, $(t_1,t_2,t_3)$ is the largest symmetric subsequence containing $t_1$ in $T$. For the rest of the sequence, as the subsequence $(t_4, t_n)$ is the same as $s_3, s_n$, the remainder of the proof of Lemma \ref{lemma: special sequence} holds.

    Truncating $T$ at $T_{n-1}$ gives a sequence of length congruent to $0 \mod 7$ and contains $\frac{3}{7}(k-7) + m_{2,7}$ symmetric subsequences.

    For $n = 2 \mod 7$, let $k$ be the smallest multiple of $7$ less than or equal to $n$, and construct the sequence 
        $$U = \sum_{i=1}^k s^* + (0,1).$$

    The addition of $0$ has no affect on the symmetric runs from Lemma \ref{lemma: special sequence}, but the addition of $1$ must be examined. As only the positions $i=2 \mod 7$ is the beginning of a symmetric subsequence in $U$ that starts with $1$, we only need to consider $U_i$ where $i=2 \mod 7$. We have $(U_{i},U_{i+1})=(1,1)$ and $(U_{n-1}, U_{n})=(0,1)$. The sequence $U$ gives a sequence of length congruent to $2 \mod 7$ and contains $\frac{3}{7}(k-7) + m_{2,2}$ symmetric subsequences.
    
    For $n = 4 \mod 7$, we use $s^{**}=(0,0,1,0,1,1,1)$, and let $k$ be the smallest multiple of $7$ greater than or equal to $n$, and construct the sequence

    $$V = \sum_{i=1}^k s^{**}$$

    Truncating $V$ after $V_{n-3}$ gives a sequence of length congruent to $4 \mod 7$ and contains $\frac{3}{7}(k-7) + m_{2,4}$ symmetric subsequences.

\end{proof}

Therefore, $m_{2,n} = \lfloor \frac{3n+10}{7} \rfloor.$

\end{document}
